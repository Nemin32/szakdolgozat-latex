% A4 lap méret, 12 pontos betűméret

% szükséges csomagok
\usepackage[utf8]{inputenc}
\usepackage[T1]{fontenc}
\usepackage[magyar]{babel}
\usepackage{float}
\usepackage{graphicx}
\usepackage{pdfpages}
% \usepackage{caption}
\usepackage[colorlinks=false, pdfborder={0 0 0}, linkcolor=black, unicode]{hyperref}
\usepackage{mathtools}
\usepackage{relsize}
\usepackage{amsmath}
\usepackage{amsfonts}
\usepackage{amssymb}
\usepackage{listings}
\usepackage{csquotes}
\usepackage{enumitem}
\usepackage{url}
\usepackage[nottoc,numbib]{tocbibind}
\usepackage{titlesec}
\usepackage{titletoc}

\usepackage{parskip}
\usepackage{tabularx}
\usepackage[format=plain, font=normalfont]{caption}
\usepackage{listings}
\usepackage{dirtytalk}

  % Graciously taken from: https://tex.stackexchange.com/a/574950
\lstdefinestyle{mystyle}{
  basicstyle=\ttfamily\smaller,
  keepspaces=true,
  breaklines=true,
  breakindent=1em,
  captionpos=b,
  linewidth=\textwidth,
  inputencoding = utf8,  % Input encoding
  extendedchars = true,  % Extended ASCII
  literate      =        % Support additional characters
  {á}{{\'a}}1  {é}{{\'e}}1  {í}{{\'i}}1 {ó}{{\'o}}1  {ú}{{\'u}}1
  {Á}{{\'A}}1  {É}{{\'E}}1  {Í}{{\'I}}1 {Ó}{{\'O}}1  {Ú}{{\'U}}1
  {à}{{\`a}}1  {è}{{\`e}}1  {ì}{{\`i}}1 {ò}{{\`o}}1  {ù}{{\`u}}1
  {À}{{\`A}}1  {È}{{\`E}}1  {Ì}{{\`I}}1 {Ò}{{\`O}}1  {Ù}{{\`U}}1
  {ä}{{\"a}}1  {ë}{{\"e}}1  {ï}{{\"i}}1 {ö}{{\"o}}1  {ü}{{\"u}}1
  {Ä}{{\"A}}1  {Ë}{{\"E}}1  {Ï}{{\"I}}1 {Ö}{{\"O}}1  {Ü}{{\"U}}1
  {â}{{\^a}}1  {ê}{{\^e}}1  {î}{{\^i}}1 {ô}{{\^o}}1  {û}{{\^u}}1
  {Â}{{\^A}}1  {Ê}{{\^E}}1  {Î}{{\^I}}1 {Ô}{{\^O}}1  {Û}{{\^U}}1
  {œ}{{\oe}}1  {Œ}{{\OE}}1  {æ}{{\ae}}1 {Æ}{{\AE}}1  {ß}{{\ss}}1
  {ẞ}{{\SS}}1  {ç}{{\c{c}}}1 {Ç}{{\c{C}}}1 {ø}{{\o}}1  {Ø}{{\O}}1
  {å}{{\aa}}1  {Å}{{\AA}}1  {ã}{{\~a}}1  {õ}{{\~o}}1 {Ã}{{\~A}}1
  {Õ}{{\~O}}1  {ñ}{{\~n}}1  {Ñ}{{\~N}}1  {¿}{{?`}}1  {¡}{{!`}}1
  {°}{{\textdegree}}1 {º}{{\textordmasculine}}1 {ª}{{\textordfeminine}}1
  {£}{{\pounds}}1  {©}{{\copyright}}1  {®}{{\textregistered}}1
  {«}{{\guillemotleft}}1  {»}{{\guillemotright}}1  {Ð}{{\DH}}1  {ð}{{\dh}}1
  {Ý}{{\'Y}}1    {ý}{{\'y}}1    {Þ}{{\TH}}1    {þ}{{\th}}1    {Ă}{{\u{A}}}1
  {ă}{{\u{a}}}1  {Ą}{{\k{A}}}1  {ą}{{\k{a}}}1  {Ć}{{\'C}}1    {ć}{{\'c}}1
  {Č}{{\v{C}}}1  {č}{{\v{c}}}1  {Ď}{{\v{D}}}1  {ď}{{\v{d}}}1  {Đ}{{\DJ}}1
  {đ}{{\dj}}1    {Ė}{{\.{E}}}1  {ė}{{\.{e}}}1  {Ę}{{\k{E}}}1  {ę}{{\k{e}}}1
  {Ě}{{\v{E}}}1  {ě}{{\v{e}}}1  {Ğ}{{\u{G}}}1  {ğ}{{\u{g}}}1  {Ĩ}{{\~I}}1
  {ĩ}{{\~\i}}1   {Į}{{\k{I}}}1  {į}{{\k{i}}}1  {İ}{{\.{I}}}1  {ı}{{\i}}1
  {Ĺ}{{\'L}}1    {ĺ}{{\'l}}1    {Ľ}{{\v{L}}}1  {ľ}{{\v{l}}}1  {Ł}{{\L{}}}1
  {ł}{{\l{}}}1   {Ń}{{\'N}}1    {ń}{{\'n}}1    {Ň}{{\v{N}}}1  {ň}{{\v{n}}}1
  {Ő}{{\H{O}}}1  {ő}{{\H{o}}}1  {Ŕ}{{\'{R}}}1  {ŕ}{{\'{r}}}1  {Ř}{{\v{R}}}1
  {ř}{{\v{r}}}1  {Ś}{{\'S}}1    {ś}{{\'s}}1    {Ş}{{\c{S}}}1  {ş}{{\c{s}}}1
  {Š}{{\v{S}}}1  {š}{{\v{s}}}1  {Ť}{{\v{T}}}1  {ť}{{\v{t}}}1  {Ũ}{{\~U}}1
  {ũ}{{\~u}}1    {Ū}{{\={U}}}1  {ū}{{\={u}}}1  {Ů}{{\r{U}}}1  {ů}{{\r{u}}}1
  {Ű}{{\H{U}}}1  {ű}{{\H{u}}}1  {Ų}{{\k{U}}}1  {ų}{{\k{u}}}1  {Ź}{{\'Z}}1
  {ź}{{\'z}}1    {Ż}{{\.Z}}1    {ż}{{\.z}}1    {Ž}{{\v{Z}}}1
}

\lstset{style=mystyle}

\AtBeginDocument{
\renewcommand{\thelstlisting}{\arabic{section}.\arabic{lstlisting}}
\renewcommand{\lstlistingname}{Forráskód}
\captionsetup[lstlisting]{labelsep=colon}
}

\lstnewenvironment{code}[2]
  {\lstset{
    firstnumber=1,
    numbers=left,
    numbersep=5pt,
    numberstyle=\footnotesize,
    xleftmargin=1.5em,
    caption={#1} \label{#2},
    alsoletter={()[].=},
    }
    \minipage[t]{\linewidth}\vspace{-14pt}
    }
  {\endminipage}

\lstnewenvironment{ebnf}[0]{
  \lstset{
    nolol=true,
    % caption={#1} \label{ebnf:#2},
    alsoletter={()[].=},
    }
    \minipage[t]{\linewidth}\vspace{-14pt}
}{\endminipage}


% times new roman betűtípus
\usepackage{times}

% lista elemek közti hely, és vonal mint lista elem előtti jel beállítása
% \setlist{nosep, label={--}}

% tartalomjegyzék formázása
\dottedcontents{section}[1.5em]{}{1.5em}{1pc}
\dottedcontents{subsection}[3em]{}{2em}{1pc}
\dottedcontents{subsubsection}[5em]{}{3em}{1pc}

% fejezet cím formázás: 14 pontos betűméret, nagybetűs
\titleformat{\section}{\normalfont\fontsize{16}{16}\mdseries\MakeUppercase}{\thesection}{1em}{}
\titleformat{\subsection}{\normalfont\fontsize{14}{14}\mdseries}{\thesubsection}{1em}{}
\titleformat{\subsubsection}{\normalfont\fontsize{14}{14}\mdseries}{\thesubsubsection}{1em}{}
\titleformat{\paragraph}{\normalfont\fontsize{12}{12}\mdseries}{\theparagraph}{1em}{}

% fejezet cím térközök
\titlespacing*{\section}{0em}{0em}{1em}
\titlespacing*{\subsection}{0em}{1em}{1em}
\titlespacing*{\subsubsection}{0em}{1em}{1em}
\titlespacing*{\paragraph}{0em}{1em}{1em}

\setcounter{secnumdepth}{5}

\setcounter{tocdepth}{2}

% margó
\usepackage[right=2.50cm, 
            left=3.50cm, 
            top=2.50cm, 
            bottom=4.00cm
            ]{geometry}

% sorköz
\linespread{1.15}

\setlength{\footskip}{2cm}

% ábra és táblázat számozás beállítása fejezetenként
\setcounter{figure}{0}
\renewcommand{\thefigure}{\arabic{section}.\arabic{figure}}
\setcounter{table}{0}
\renewcommand{\thetable}{\arabic{section}.\arabic{table}}
\numberwithin{equation}{section}
\numberwithin{figure}{section}

% ábra, táblázat cím formázás
% \captionsetup[figure]{labelfont={it, small},textfont={it, small},labelsep=colon}
% \captionsetup[table]{labelfont={it, small},textfont={it, small},labelsep=colon}

\captionsetup[figure]{labelsep=colon}
\captionsetup[table]{labelsep=colon}

% hivatkozások formátuma és bib fájl importálása
\nocite{*}

\usepackage[backend=biber,style=ieee,doi=true,isbn=true,url=false,eprint=false,sorting=none]{biblatex}
\addbibresource{references.bib}


% bekezdés behúzása
% \setlength{\parindent}{5 mm}

% bekezdés térköz
\setlength{\parskip}{8pt}

\newcommand{\addimage}[3]{
  \begin{figure}[H]
    \centering
    \includegraphics[width=0.75\linewidth]{img/#1}
    \caption{#2}
    \label{fig:#3}
  \end{figure}
}

% ábra és táblajegyzék formázása
\makeatletter
\renewcommand\listoffigures{
    \section{\listfigurename}
      \@mkboth{\MakeUppercase\listfigurename}%
              {\MakeUppercase\listfigurename}%
    \@starttoc{lof}%
    }

\renewcommand\listoftables{
    \section{\listtablename}
      \@mkboth{\MakeUppercase\listtablename}%
              {\MakeUppercase\listtablename}%
    \@starttoc{lot}%
    }
  

\renewcommand\lstlistoflistings{
    \section{\lstlistlistingname}
      \@mkboth{\MakeUppercase\lstlistlistingname}%
              {\MakeUppercase\lstlistlistingname}%
    \@starttoc{lol}%
    }
\makeatother
