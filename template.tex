% A4 lap méret, 12 pontos betűméret

% szükséges csomagok
\usepackage[utf8]{inputenc}
\usepackage[T1]{fontenc}
\usepackage[magyar]{babel}
\usepackage{float}
\usepackage{graphicx}
\usepackage{pdfpages}
% \usepackage{caption}
\usepackage[colorlinks=false, pdfborder={0 0 0}, linkcolor=black, unicode]{hyperref}
\usepackage{mathtools}
\usepackage{relsize}
\usepackage{amsmath}
\usepackage{amsfonts}
\usepackage{amssymb}
\usepackage{listings}
\usepackage{csquotes}
\usepackage{enumitem}
\usepackage{url}
\usepackage[nottoc,numbib]{tocbibind}
\usepackage{titlesec}
\usepackage{titletoc}

\usepackage{parskip}
\usepackage{tabularx}
\usepackage[format=plain, font=normalfont]{caption}


% times new roman betűtípus
\usepackage{times}

% lista elemek közti hely, és vonal mint lista elem előtti jel beállítása
% \setlist{nosep, label={--}}

% tartalomjegyzék formázása
\dottedcontents{section}[1.5em]{}{1.5em}{1pc}
\dottedcontents{subsection}[3em]{}{2em}{1pc}
\dottedcontents{subsubsection}[5em]{}{3em}{1pc}

% fejezet cím formázás: 14 pontos betűméret, nagybetűs
\titleformat{\section}{\normalfont\fontsize{16}{16}\mdseries\MakeUppercase}{\thesection}{1em}{}
\titleformat{\subsection}{\normalfont\fontsize{14}{14}\mdseries}{\thesubsection}{1em}{}
\titleformat{\subsubsection}{\normalfont\fontsize{14}{14}\mdseries}{\thesubsubsection}{1em}{}
\titleformat{\paragraph}{\normalfont\fontsize{12}{12}\mdseries}{\theparagraph}{1em}{}

% fejezet cím térközök
\titlespacing*{\section}{0em}{0em}{1em}
\titlespacing*{\subsection}{0em}{1em}{1em}
\titlespacing*{\subsubsection}{0em}{1em}{1em}
\titlespacing*{\paragraph}{0em}{1em}{1em}

\setcounter{secnumdepth}{5}


% margó
\usepackage[right=2.50cm, 
            left=3.50cm, 
            top=2.50cm, 
            bottom=4.00cm
            ]{geometry}

% sorköz
\linespread{1.15}


% ábra és táblázat számozás beállítása fejezetenként
\setcounter{figure}{0}
\renewcommand{\thefigure}{\arabic{section}.\arabic{figure}}
\setcounter{table}{0}
\renewcommand{\thetable}{\arabic{section}.\arabic{table}}
\numberwithin{equation}{section}
\numberwithin{figure}{section}

% ábra, táblázat cím formázás
% \captionsetup[figure]{labelfont={it, small},textfont={it, small},labelsep=colon}
% \captionsetup[table]{labelfont={it, small},textfont={it, small},labelsep=colon}

\captionsetup[figure]{labelsep=colon}
\captionsetup[table]{labelsep=colon}

% hivatkozások formátuma és bib fájl importálása
\nocite{*}

\usepackage[backend=biber,style=ieee,doi=true,isbn=true,url=false,eprint=false]{biblatex}
\addbibresource{references.bib}


% bekezdés behúzása
% \setlength{\parindent}{5 mm}

% bekezdés térköz
\setlength{\parskip}{8pt}

\newcommand{\addimage}[3]{
  \begin{figure}[H]
    \centering
    \includegraphics[width=0.75\linewidth]{img/#1}
    \caption{#2}
    \label{fig:#3}
  \end{figure}
}

% ábra és táblajegyzék formázása
\makeatletter
\renewcommand\listoffigures{
    \section{\listfigurename}
      \@mkboth{\MakeUppercase\listfigurename}%
              {\MakeUppercase\listfigurename}%
    \@starttoc{lof}%
    }

\renewcommand\listoftables{
    \section{\listtablename}
      \@mkboth{\MakeUppercase\listtablename}%
              {\MakeUppercase\listtablename}%
    \@starttoc{lot}%
    }
\makeatother
