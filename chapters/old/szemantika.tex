\subsection{Üres helyek és sorkihagyások}

A Pythonhoz hasonló nyelvekkel ellentétben, a Pszeudokód nem veszi figyelembe az üres helyeket, ezeket csupán az átláthatósághoz és a kód fordításánál használjuk fel, a nyelvi elemek és utasítások elhatárolására. A végső szintaxisfából kiszűrésre kerülnek.

Mivel a kód nem tartalmaz semmiféle sorlezáró jelet (ilyen például a C-hez hasonlatos nyelvekben használt pontosvessző), így az egymás után következő utasításokat célszerűen sortöréssel választjuk el egymástól. A kód elemeit pedig egy soron belül tetszőleges számú, de legalább egy üres hellyel választhatjuk el.

\subsection{Kommentek}

\begin{footnotesize}
\begin{verbatim}
<comment> ::= '//' .* '\n'
\end{verbatim}
\end{footnotesize}

Bár a jegyzetben használt Pszeudokód nem definiálja a kommentelést, mint lehetséges nyelvi elem, hasznossága miatt ez az implementáció figyelmen kívül hagyja a C nyelvből megszokott két perjellel kezdődő sorokat.

\subsection{Típusok}

\begin{footnotesize}
\begin{verbatim}
<type> ::= ('egész' | 'logikai' 'szöveg') 'tömb'?
\end{verbatim}
\end{footnotesize}

A Pszeudokód háromféle típust különböztet meg. Ezek az egész számok (az értelmező lebegőpontos értékeket nem támogat), a logikai értékek, és a karakterláncok. Ezen kívül a nyelv értelmezi még az ezen típusokból alkotott tömböket is, melyet a "tömb" kulcsszó hozzáillesztésével jelölünk.

\subsection{Atomi értékek}

\begin{footnotesize}
\begin{verbatim}
<number> ::= '0' | '1'-'9' '0'-'9'*
<bool> ::= 'igaz' | 'hamis' | 'Igaz' | 'Hamis'
<string> ::= '"' .* '"'

<atom> ::= <number> | <bool> | <string>
\end{verbatim}
\end{footnotesize}

Az atom kifejezést a Scheme nyelvből vettem át. Olyan értékeket jelöl, melyek minden külső tényezőt figyelmen kívül hagyva, mindig önmagukra értékelődnek ki. A Pszeudokód három ilyen értéket ismer, ezek az egész számok, az idézőjelek közé zárt tetszőleges hosszúságú karakterláncok, és az igaz és hamis logikai értékek. Az utóbbi esetén a kulcsszavak kis és nagybetűvel is értelmezve vannak.

Tehát az '500' karakterlánc az ötszázas számnak felel meg, az 'igaz' az igaz logikai értéknek, a '"példa"' pedig a "példa" szövegnek.

\subsection{Műveletek}

\begin{footnotesize}
\begin{verbatim}
<value> ::= <array_index> | <variable> | <atom>
<factor> ::= <value> | '(' <logic_binop> ')'
\end{verbatim}
\end{footnotesize}

\subsubsection{Aritmetika}

\begin{footnotesize}
\begin{verbatim}
<arithm_add_op> ::= '+' | '-'
<arithm_mul_op> ::= '*' | '/' | 'mod'

<arithm_add_binop> ::= <expression> <arithm_add_op> <expression>
<arithm_mul_binop> ::= 
  | <expression> <arithm_mul_op> <expression> 
  | <arithm_add_binop>
\end{verbatim}
\end{footnotesize}

\subsubsection{Összehasonlítások}

\begin{footnotesize}
\begin{verbatim}
<comparison_op> ::= '<=' | '>=' | '<' | '>' | '=' | '=/='
<comparsion_binop> ::= 
  | <expression> <comparison_op> <expression>
  | <arithm_mul_binop>
\end{verbatim}
\end{footnotesize}

\subsubsection{Logikai}

\begin{footnotesize}
\begin{verbatim}
<logic_op> ::= 'és' | 'vagy'
<logic_binop> ::= 
  | <expression> <logic_op> <expression> 
  | <comparsion_binop>
\end{verbatim}
\end{footnotesize}


A műveletek szabadon ágyazhatóak egymásba, ezáltal lehetővé válnak a C-szerű nyelvekből megszokott komplex összehasonlítások, mint például annak ellenőrzése, hogy egy lista végére értünk-e már és, ha nem, akkor az éppen ellenőrzött elem milyen értékkel rendelkezik.

\subsubsection{Negálás}

\begin{footnotesize}
\begin{verbatim}
<negation> ::= '~' <expression>
\end{verbatim}
\end{footnotesize}

Lehetőségünk van még egy logikai értéket negálni is, ez esetben az igaz érték hamissá válik, a hamis pedig igazzá.

\subsection{Változók}

\begin{footnotesize}
\begin{verbatim}
<variable> ::= 'a'-'z'
\end{verbatim}
\end{footnotesize}

A változók nevei csupa kisbetűvel írt szavak. Típusuk értékadáskor dől el implicit módon, a jobboldalt álló érték típusa alapján. Egy változó típusa inicializálás után nem változhat. Konstans deklaráció nem lehetséges, a programozó feladata, hogy a változók értéke ne változhasson, ha az nem szándékos.

A nyelv elődeklarációt nem támogat, a változók létrehozáskor kötelezően értéket is kapnak, ebből következően null-értékkel rendelkező vagy üres változó nem létezhet.

A változó értékét annak nevével érhetjük el. Nem létező változó értékének elérése típusellenörzési vagy futásközbeni hibához vezet.

\subsubsection{Értékadás}

\begin{footnotesize}
\begin{verbatim}
<assignment> ::= <variable> '<-' <expression>
\end{verbatim}
\end{footnotesize}

Egy változó bármely adattípust képes eltárolni, ám típusa nem változhat meg (példa: ha egy számot rendelünk egy változóhoz, onnantól nem rendelhetünk hozzá szöveget). Az értékadáshoz a '<-' (nyíl) operátort használjuk. Ha egy változó értékét egy másikára állítjuk érték-alapú másolás történik, tehát az így keletkezett adat az eredetitől függetlenül módosítható.

\subsubsection{Tömbök}

\begin{footnotesize}
\begin{verbatim}
<array_create> ::= <variable> '<-' 'Létrehoz[' <type> '](' <expression> ')'
<array_index> ::= <variable> '[' <expression> (',' <expression>)* ']'
<array_set> ::= <variable '[' <expression> ']' '<-' <expression>
<array_shorthand> ::= '(' <expression> (',' <expression>)* ')'
\end{verbatim}
\end{footnotesize}

Tömböket kétféleképp deklarálhatunk a Pszeudokód nyelvben.

Egyrészt készítetünk megadott típussal és hosszal rendelkező tömböt, mely ekkor a típusnak megfelelő üres értékekkel töltődik fel (0 szám, "" szöveg, és hamis logikai érték esetén). 

Másrészt lehetőségünk van kézzel megadni a tömb elemeit. Ezesetben a típusoknak nem feltétlen kell egyezniük. Tömbök értékeinek lekérése esetén nem a C-ből megszokott \verb|tömb[x][y]| szintaxist használjuk, hanem ehelyett a C\#-ból ismert \verb|tömb[x, y]|-t.

\subsubsection{Cím lekérése}

\begin{footnotesize}
\begin{verbatim}
<reference> ::= '&' (<array_index> | <variable>)
\end{verbatim}
\end{footnotesize}

Lehetőségünk van még egy változó címének lekérésére is. Ez egy olyan számérték, mely a nyelvet futtató virtuális gép memóriájának a változó értékének megfelelő cellájára mutat. Ez a tömbök elemeinek címét is képes lekérni.

A függvények címszerint átadott paramétereinél kerül felhasználásra. Tömbökkel egyszerű mutató-aritmetikát is végezhetünk (tömb + n = n. elem), viszont a futtatási környezet hibát dob, ha megpróbálunk a tömb határain kívülre indexelni ilyen módon.

\subsection{Folyamatvezérlési szerkezetek}

A Pszeudokód a három leggyakoribb Folyamatvezérlési szerkezetet bocsájtja a programozó rendelkezésére.

\subsubsection{Elágazás}

\begin{footnotesize}
\begin{verbatim}
<else> ::= 'különben' <block>
<else_if> ::= 'különben' 'ha' <expression> 'akkor' <block>
<if> ::= 'ha' <expression> 'akkor' <block> (<else>? | <else_if>+ <else>)
\end{verbatim}
\end{footnotesize}

A nyelv háromféle elágazást támogat: 

\begin{itemize}
  \item Egyszerű predikátum-vizsgálat (if)
  \item Predikátum-vizsgálat és egy alternatív ág (if-else)
  \item Predikátum-vizsgálat, egy/több alternatív ág, és egy végleges ág (if-egy/több else if-else)
\end{itemize}

\subsubsection{Ciklusok}

\begin{footnotesize}
\begin{verbatim}
<while> ::= 'ciklus' 'amíg' <expression> <block> 'ciklus' 'vége'
<do_while> ::= 'ciklus' <block> 'amíg' <expression>
<for> ::= 
  'ciklus' <variable> '<-' 
  <expression> ('-tól'|'-től') <expression> '-ig'
    <block> 
  'ciklus' 'vége'
\end{verbatim}
\end{footnotesize}

A Pszeudokód a C nyelvből ismert for, while, és do while ciklusszerkezetek megfelelőit támogatja. For-szerű ciklus esetén a változó az első értékkel inicializálódik és ciklusonként egyel nő, amíg el nem éri a második értéket. Az érték elérésekor még lefut a ciklus. GOTO-nak megfelelő utasítás nincs, ahogy Assembly/C-szerű címkék meghatározására sincs lehetőség.

\subsubsection{Függvények}

\begin{footnotesize}
\begin{verbatim}
<function_name> ::= 'A'-'Z' ('a'-'z' | 'A'-'Z')*
\end{verbatim}
\end{footnotesize}

Lehetőségünk van függvények létrehozására és hívására is. A függvények neveit nagybetűvel kezdjük, így különböztetve meg őket a változóktól.

\paragraph{Deklarálás}

\begin{footnotesize}
\begin{verbatim}
<func_decl_parameter> ::= 'címszerint'? <variable> ':' <type>

<func_decl_param_list> ::= 
  | '(' ')' 
  | '(' <func_decl_parameter> (',' <func_decl_parameter>)* ')'

<function_declaration> ::= 
  'függvény' <function_name> <func_decl_param_list> 
    <block> 
  'függvény' 'vége'
\end{verbatim}
\end{footnotesize}

A függvények deklarációját azok meghívása előtt kell megtennünk. Új függvény másik függvényen belül nem deklarálható. Lehetőségünk van érték szerint és cím szerint is bekérni paramétereket.

Előbbi esetben a változó értékének egy másolata kerül átadásra, melynek változtatása nem befolyásolja az eredeti értéket. Cím szerinti átadás esetén viszont a két változó ugyanarra az értékre mutat, így az egyik változtatása változtatja a másikat is.

\paragraph{Visszatérés}

\begin{footnotesize}
\begin{verbatim}
<return> ::= 'vissza' <expression>
\end{verbatim}
\end{footnotesize}

Függvényból értékkel visszatérni a 'vissza' kulcsszó segítségével tudunk. Ennek meghívásakor a függvény azonnal megszakad és a kapott értékkel tér vissza.

\paragraph{Hívás}

\begin{footnotesize}
\begin{verbatim}
<func_call_parameter_list> ::= 
  | '(' ')' 
  | '(' <expression> (',' <expression>)* ')'
<function_call> ::= <function_name> <func_call_parameter_list>
\end{verbatim}
\end{footnotesize}

Függvény hívható kifejezésként is, mely esetben az azt hívó utasítás a visszatérési értékét kapja meg, vagy hívható önmagában utasításként is, ekkor a visszatérési érték eldobásra kerül. Függvény hívhat más függvényt.

\subsection{Kiírás}

\begin{footnotesize}
\begin{verbatim}
<print> ::= 'kiír' <expression>
\end{verbatim}
\end{footnotesize}

Meghíváskor a kapott érték a felhasználó számára látható módon megjelenítésre kerül (például konzolon).

\subsection{Megszakítás}

\begin{footnotesize}
\begin{verbatim}
<debug> ::= 'debug'
\end{verbatim}
\end{footnotesize}

A kulcsszó használatakor az értelmező futtatása rögtön megszakad. Ezután a felhasználó a kódot léptetheti vagy a változók és a verem értékeit megvizsgálhatja.

\subsection{Kifejezések}

\begin{footnotesize}
\begin{verbatim}
<expression> ::= <negation> | <array_shorthand> | <function_call> | <logic_binop>
\end{verbatim}
\end{footnotesize}

A legegyszerűbb kifejezések az úgynevezett atomok, melyek olyan értékek, amik saját magukká válnak kiértékelés után. Ilyenek például a számok. Az „5” szám atom, hisz az értéke önmaga, vagyis az ötös szám. A szövegek és a logikai értékek is ide tartoznak.

Ezen alap építőkövekre épülnek a komplexebb kifejezések, például a számítások és logikai műveletek, melyek rekurzívan egymásra is épülhetnek (például többszörös összeadás.) 

Ide tartozik még a függvényhívás is, viszont mivel eljárások esetében nem térünk vissza értékekkel, így csak a kód egyszerűbbé tétele miatt soroljuk a kifejezésekhez.

\subsection{Utasítások}

\begin{footnotesize}
\begin{verbatim}
<statement> ::= 
  | <array_set> 
  | <array_create> 
  | <if> 
  | <return> 
  | <for> 
  | <assignment> 
  | <function_declaration> 
  | <print> 
  | <do_while> 
  | <while> 
  | <debug> 
  | <function_call>
<block> ::= <statement>*
\end{verbatim}
\end{footnotesize}

Míg a kifejezések magukat az elvégzett számolásokat és műveleteket definiálják, az utasítások a program folyamatvezérlését, a változók elérését és módosítását, a standard kimenetre történő írást, és a hibakereső töréspontjait határozzák meg.

Bár az utasítások definíció szerint nem térhetnek vissza értékekkel, ez alól kivételt képez a „vissza” utasítás, mely annak ellenére, hogy ehhez a kategóriához van sorolva, képes egy függvényt megszakítani és egy értékkel visszatérni.

A program szintaxisfájának gyökere egy darab \texttt{<block>} elem.