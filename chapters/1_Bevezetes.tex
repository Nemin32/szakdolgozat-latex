A szakdolgozatom témája egy az egyetemen tanított Pszeudokód értelmezésére és futtatására képes program, mely a beírt kód futtatására, léptetésére, az algoritmust futtató környezet belső információnak (változók, vermek és memória tartalma) kiíratására, és az algoritmusok végeredményének a kimenetre való kiírására.

A döntésem azért erre esett, mivel érdekelnek a programozási nyelvek és fordítóik működése és mivel az első félévben a programozás tárgy legnagyobb nehézségét a Pszeudokód megértése okozta. Épp ezért döntöttem úgy, hogy olyasmit készítenék, ami kapcsolódik az érdeklődési körömhöz és reális haszonnal is rendelkezik, hisz egy ilyen program a későbbiekben segítséget nyújthatna a gólyáknak az algoritmusok megértésében.