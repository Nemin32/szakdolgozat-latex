A szakdolgozatom témája egy, az egyetemen tanított Pszeudokód\cite{jegyzet}\cite{jegyzet2} értelmezésére képes program. Ez egy olyan értelmező, mely képes a beírt kódot futtatható alakba hozni, ezt az átalakított formátumot le is futtatni, a futtatást tetszőleges ponton megállítani és a programot az egyes utasításai között léptetni, kiírni az algoritmust futtató környezet belső információit (változók, vermek és memória tartalma) és az algoritmusok végeredményét a felhasználói kimeneten megjeleníteni.

A választásom azért erre a témára esett, mert régóta érdekelnek a programozási nyelvek és fordítóik működése és mert az első félévben a Szoftvertervezés és Fejlesztés I. tárgy legnagyobb nehézségét éppen a Pszeudokód megértése okozta számomra.

Épp ezért döntöttem úgy, hogy olyasmit készítenék, ami kapcsolódik az érdeklődési körömhöz és reális haszonnal is rendelkezik, hisz egy ilyen program a későbbiekben segítséget nyújthatna a elsőéves hallgatóknak az algoritmusok megértésében.

A dolgozat keretei között először is specifikálom a Pszeudokód nyelv elemeit, szintaxisát és működését. Ezután röviden bemutatom azokat az alapvető technikákat, amelyekből a programom felépül. Ezt követi pár hasonló program leírása, majd annak mérlegelése, hogy melyik programozási nyelvet is szeretném használni.

Ezt a program rendszerterve követi, mely nagy vonalakban specifikálja annak felépítését és a felhasználó által elérhető interfészét. Ezután következik a megvalósítás részletes leírása és annak tesztelése. Végül a további esetleges fejlesztések felsorolása és a szakdolgozat összegzése.
