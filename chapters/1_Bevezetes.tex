A szakdolgozatom témája egy, az egyetemen tanított Pszeudokód\cite{jegyzet}\cite{jegyzet2} értelmezésére és futtatására képes program, mely a beírt kód futtatására, léptetésére, az algoritmust futtató környezet belső információnak (változók, vermek és memória tartalma) kiíratására, és az algoritmusok végeredményének a kimenetre való kiírására.

A döntésem azért erre esett, mivel érdekelnek a programozási nyelvek és fordítóik működése és mivel az első félévben a programozás tárgy legnagyobb nehézségét a Pszeudokód megértése okozta.

Épp ezért döntöttem úgy, hogy olyasmit készítenék, ami kapcsolódik az érdeklődési körömhöz és reális haszonnal is rendelkezik, hisz egy ilyen program a későbbiekben segítséget nyújthatna a elsőéves hallgatóknak az algoritmusok megértésében.

A dolgozat először is specifikálja a Pszeudokód nyelv elemeit, szintaxisát, és működését. Ezután röviden bemutatom azokat az alapvető technikákat, amelyekből a programom felépül. Ezt követi pár hasonló program összegzése, majd annak eldöntése, hogy melyik programozási nyelvet is fogom használni.

Ezt a program rendszerterve követi, mely nagy vonalakban specifikálja annak felépítését és a felhasználó által elérhető interfészét. Ezután következik a megvalósítás részletes leírása és annak tesztelése. Végül a további esetleges fejlesztések felsorolása és a szakdolgozat összegzése.