Bár az interpreterek nem triviális programok, mégis szinte bármely programnyelv felhasználható az implementálásukra. Mivel szeretném, hogy a projektmunkám webes felületen is elérhető legyen, így ezt a funkcionalitást mindenképp figyelembe kell vennem, amikor a nyelveket értékelem.

Ideális esetben a nyelv rögtön weben is futtatható állománnyá fordul, ám mivel a legtöbb programozási nyelv nem képes ilyesmire, ezért jó lehetőségnek kínálkozik az úgy nevezett WebAssembly-re való fordítás.

„A WebAssembly (rövidítve Wasm) egy biztonságos, hordozható, és alacsony-szintű kódformátum, melyet gyors futtatásra és kompakt méretű ábrázolásra terveztek. A fő célja, hogy lehetővé tegyen gyorsan futtatható programokat a Weben, anélkül hogy bármilyen Web-irányú feltételezést tenne vagy Web-specifikus funkciókat tenne lehetővé, ezáltal más közegekben is használhatóvá váljon.”[20]

Alább szeretném kifejteni melyik nyelveket vettem fontolóra:

5.1.1 C

Természetesen nem hagyhatjuk ki az egyik legöregebb és mégis széles körűen használatos nyelvet.

A C pozitívumai közé sorolandó, hogy nagyon könnyen lehet más platformokra fordítani. Az Emscripten nevű fordítói csomag segítségével a C Wasm formátumba is fordítható, ezáltal böngészőben is futni képes.

„Az Emscripten egy nyílt-forráskódú fordító WebAssembly-re. A segítségével C és C++ kódot vagy bármely más LLVM-et használó nyelvet WebAssembly-re fordíthatunk és az Web-en, Node.js-en, vagy más wasm futtatási környezetben futtathatjuk.”[4]

A nyelv pozitívumai közé sorolandó még, hogy nem túl „meglepő” (kevés szintaxissal és aránylag átlátható szabályokkal rendelkezik), kifejezetten gyors és kis állományokká fordul, és szinte páratlan szabadságot ad a fejlesztőnek a rendszer erőforrásainak felhasználásához.

Cserébe  viszont a több évtizedes standard könyvtár és a nyelv pár furcsasága miatt könnyen futhat a programozó furcsa hibákba, mint például a szövegeknél esetlegesen hiányzó null byteok esetén történő túlcsordulások, vagy a hibásan lefoglalt / korábban felszabadított memóriaterületekre való írás/olvasás esetén előforduló végzetes hibák.

Ezen kívül a teljesen manuális memóriakezelés bár nagyon hasznos tud lenni, de jelen problémához fölösleges plusz terhet jelent csak. Ezen kívül a nyelv nem támogatja az osztályokat, helyette struct-ok átadogatásával oldjuk meg az egybefüggő objektumok megváltoztatását. Ez sajnos nagyon könnyen tud átláthatatlansághoz vezetni, mely egy ilyen nem-triviális projektnél elfogadhatatlan.

Bár a nyelv az egyik személyes kedvencem, számomra nem megfelelő a feladathoz. A nyelv értékelésének összefoglalója az 5.1-es táblázatban látható.

5.1. táblázat: A C nyelv értékelése.

Pozitívumok

Negatívumok

Könnyen fordítható Wasm-be

Manuális memóriakezelés

Jól felszerelt eszköztár

Nincs OOP

Személyes ismeretség

Kényelmi funkciók hiánya (pl.: regex, for-each, map, adatszerkezetek, stb.)

Más nyelveket is ebben írtak (pl.: Python)

Nincs könnyű mód a hibakezelésre (segfault)



5.1.2 C++

A C++ kijavítja a C főbb hibáit a RAII-vel (Resource Aquisition Is Initialization[18], „az erőforrás lefoglalása egyben inicializáció is,” magyarul a különböző adattagok létrehozásakor a memória automatikusan lefoglalódik, megszűnésükkor pedig felszabadul) és egy sokkal kevésbé meglepő standard könyvtárral, viszont cserébe új negatívumokat ad az egyenlethez. 

Bár az Emscripten képes Wasm formátumba fordítani a nyelvet, a hibakezelése problematikus. Alapesetben a generált kód végzetes hibával lép ki bármilyen hiba esetén, beleértve az amúgy elkapható exception-öket. Ugyan egy kapcsoló segítségével engedélyezhetjük, hogy az utóbbi hibák újra elkaphatóak legyenek, ez súlyos lelassuláshoz vezet a programban.[5]

Ezen kívül a szövegkezelése a lentebb tárgyalt nyelvekhez képest igen nehézkes. Bár vannak regex könyvtárak, de ezekkel kapcsolatban semmi ismeretségem nincs. A C++ rendelkezik hibakezeléssel, viszont ennek ellenére dobhat segfaultot1. A nyelv nem rendelkezik hivatalos csomagkezelővel sem, így a könyvtárak beimportálása nem feltétlen egyszerű és nem garantált, hogy ezek valóban át is fordulnának Wasm-ra.

Mivel a nyelv használhatósága ennyire kétséges, így nem felel meg számomra a feladathoz. A nyelv értékelésének összefoglalója az 5.2-es táblázatban látható.

5.2. táblázat: A C++ nyelv értékelése.

Pozitívum

Negatívum

Van OOP

Bár van hibakezelés, de segfaultok itt is történhetnek

RAII (memóriakezelést leegyszerűsíti)

A szövegkezelés nehézkes (stream-ek, atoi és hasonló átalakító függvények furcsaságai)

Ésszerű, kiforrott standard könyvtár

A WebAssemblybe való fordítás lassú és több előkészületet igényel

Személyes tapasztalat (bár nem annyi mint a C-vel)

Nem minden külső könyvtár alkalmas a Wasm-ba való fordításra



5.1.3 C\# 

„A Blazor a Microsoft új kísérletbeli framework-je, mely pluginok használata nélkül hozza el a C\#-ot bármely böngészőbe.”[11] Az előző két nyelvvel ellentétben a C\# teljesen automatikus memóriakezeléssel rendelkezik, ezen kívül a C++-hoz hasonlóan ugyanúgy lehet benne osztályokat és interfészeket használni, mely által a kód sokkal átláthatóbb.

Viszont míg a többi nyelvnél a kliens maga képes a kód teljes feldolgozására és lefuttatására, a Blazor használatánál szükséges egy szervergép is, melyet fel kell szerelnünk a szerver futtatásának minden szükséges elemével (megfelelő hardver, .NET Core telepítése.)

Bár az egyetemi tanulmányaim miatt sok tapasztalattal rendelkezem már a nyelvvel kapcsolatban, nem szívesen kötném magam egy webes frameworkhöz, így a C\# sem megfelelő számomra a feladathoz. A nyelv értékelésének összefoglalója az 5.3-as táblázatban látható.

5.3. táblázat: A C\# nyelv értékelése.

Pozitívum

Negatívum

Sok személyes tapasztalat

Terjengősebb mint a többi nyelv

Erős OOP

Semennyi tapasztalatom nincs a webes frameworkökkel

A memóriakezelést a rendszer intézi, nekünk nem kell foglalkoznunk vele

Futtatási környezetet igényel, nagyobb mint a fentebbi nyelvek.

Kiváló standard könyvtár



Blazor framework kifejezetten webes alkalmazások fejlesztésére



Jó szövegkezelés





5.1.4 TypeScript

„A TypeScript egy erősen típusos programozási nyelv, mely a JavaScriptre épül és bármely kódméret esetén jobb eszközkészletet nyújt.”[10] Mivel nem igényel valódi fordítást, ezért nagyon gyorsan lehet vele prototípusokat gyártani és a program futtatása szinte semmi előfeltételt nem igényel egy böngészőn túl.

Viszont mivel a TypeScript alatt futó JavaScript dinamikus típusokkal rendelkezik és szinte semmi analízist nem futtat a kódon futtatás előtt, ezért könnyű hibákba futni. Ezen kívül bár rendelkezik pár OOP struktúrával, ezek nem annyira kiforrottak, mint a fentebb említett nyelvek.

Bár a nyelv nagyon közel áll ahhoz, amit keresek, az általa nyújtott garanciák nem kifejezetten hasznosak a jelenlegi projekthez, így nem akarom magam az eszköztárához kötni. A nyelv értékelésének összefoglalója az 5.4-es táblázatban látható.

5.4. táblázat: A TypeScript nyelv értékelése.

Pozitívum

Negatívum

Natívan az interneten futó kód

Gyengébb típusellenőrzés mint egy normálisan fordított nyelvben

Sok személyes tapasztalat

Lassabb mint a fentebbi nyelvek

Nagyon gyorsan prototípizálható nyelv

Mivel a TS alatt futó JS dinamikusan típusos emiatt könnyen lehet váratlan hibákba futni

Jó szövegkezelés (beépített regex)

A nyelv OOP-je nem kifejezetten kiforrott

Automatikus memóriakezelés, extra védelmek a böngészőben való futás miatt





5.1.5 LiSP

A LiSP[16] egy C-hez hasonló korú nyelv, mely lényege, hogy a kód és az adat ugyanolyan adatszerkezetben vannak tárolva. Ez nem más, mint a láncolt lista, mely lehetővé teszi, hogy bármilyen hosszúságú és bonyolultságú adatot el tudjunk tárolni aránylag egyszerű módon.

Bár egyáltalán nem tartom megfelelő nyelvnek a projektemhez, nem hagyhattam ki a megemlítését, hisz a láncolt lista futtatása nagyon hasonlít a virtuális gépről szóló fejezetben tárgyalt működéséről.

A Structure and Interpretation of Computer Programs[1] című könyv sokat segített a folyamat megértésében, mely a LiSP egyik átalakított változatát, a Scheme-t használja és egy fejezetében egy interpreter majd később egy fordító megírását mutatja be. A nyelv értékelésének összefoglalója az 5.5-ös táblázatban látható.

5.5. táblázat: A LiSP nyelv értékelése.

Pozitívum

Negatívum

A kód maga egyben adat is

Semmi reális tapasztalatom nincs a nyelvvel

Nem igényel fordítást, gyorsan lehet prototípizálni vele

Idegen, nehezen átlátható szintaxissal rendelkezik

Páratlan listafeldolgozási képességekkel rendelkezik

Nem igazán interneten át való futtatásra kitalált nyelv

Kiforrott standard könyvtárral rendelkezik

Nem szeretném használni



5.1.6 Rust

„A Rust programozási nyelv segít, hogy gyorsabb és megbízhatóbb szoftvert írjunk. A magas-szintű ergonómia és az alacsony-szintű irányítás gyakran összeegyeztethetetlen a programozási nyelvek dizájnjában; a Rust ezt az állítást kérdőjelezi meg.”[7]

A Rust egy relatíve új programozási nyelv, mely céljának a teljes memória-biztonságot és a gyakori hibák elkerülését tűzte ki, emellett natív fordításra is képes. A statikus analízise erősebb az összes fentebb felsorolt nyelvnél, a sebessége pedig a C/C++-hoz hasonlatos. Ezen kívül remek könyvtárakkal rendelkezik, melyek például a regexelést nagyon egyszerűvé teszik. A nyelv natívan fordul WebAssembly-re így ez se okoz problémát.

Cserébe a nyelv írása nagyon nehézkes, a fordító ott ahol tud felhívja a figyelmét a programozónak az esetleges hibákra vagy kifelejtett ellenőrzésekre. Ez természetesen a program minőségére nézve abszolút pozitív, ám cserébe nagyon nehézkessé teszi a prototípusok létrehozását.

Mivel a nyelv használata ilyen nehézkes és a prototípizálás lassú, így nem szeretném ezt a nyelvet választani. A nyelv értékelésének összefoglalója az 5.6-os táblázatban látható.

5.6. táblázat: A Rust nyelv értékelése.

Pozitívum

Negatívum

Gazdag standard könyvtár

Nehézkes szintaxissal rendelkezik

Jó hibakezelés, segfault átlagos kódban lehetetlen

Nincs vele elég tapasztalatom

Remek könyvtárak, csomagkezelővel

Fordítás kicsit lassú

Natív fordulás böngésző által is érthető formátumba





5.1.7 JavaScript

Mivel a program követelményei közé tartozik, hogy internetről elérhető felületen lehessen használni és mivel az ANTLR natívan képes rá kódot generálni[14], így végül a JavaScriptre esett a választásom, mint a szakdolgozat megvalósításához választott programozási nyelv.

Bár a JavaScript még a TypeScriptnél is kevesebb ígéretet tesz a típusok biztonságáról, ez nem jelent különösebb hátrányt, mivel az ANTLR által generált kód alapból se él ezekkel a típusossági garanciákkal. Habár létezik egy TypeScriptbe átültetett változata a programnak, néhány egyszerű teszt során azt tapasztaltam, hogy a kód sokkal kevésbé könnyen kezelhető, mint a sima JavaScript verzióval, így végül az utóbbi mellett döntöttem. A nyelv értékelésének összefoglalója az 5.7-es táblázatban látható.

Pozitívum

Negatívum

A legegyszerűbb szerverről felszolgálható

Lassabb mint egy fordított nyelv

A végfelhasználónak csak böngészővel kell rendelkeznie

A legtöbb rossz típusból eredő hiba csak futás során derül ki

Dinamikus tömbökkel és automatikus memóriakezeléssel rendelkezik

Kiforrott eszközökkel rendelkezik, melyek segítségével a hibák száma és a kód mérete minimalizálható

