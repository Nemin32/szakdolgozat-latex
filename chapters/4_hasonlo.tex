Az alábbi fejezetben szeretném bemutatni azokat a nyelveket és implementációjukat, melyek segítettek a téma megértésében.

\subsection{Brainf**k}

\subsubsection{A nyelvről röviden}

A Brainf**ck (BF) egy úgynevezett „ezoterikus programozási nyelv,” mely annyit jelent, hogy nem valós használatra szánták, hanem szimplán viccből vagy valami érdekes megkötés vállalásával alkották. A nyelv érdekessége, hogy összesen nyolc utasítást tartalmaz, melyek mind egy-egy karakterből állnak, ennek ellenére a nyelv Turing-teljes és elméletileg bármely más Turing-teljes nyelven megoldott probléma átültethető bele, bár ez a nyelv különleges szintaxisa miatt felettébb nehéz.

Azért választottam ennek a nyelvnek a bemutatását, mivel ez az egyik legkisebb olyan nyelv, amit nagyon egyszerűen megvalósíthatunk, de mellé megtanítja a nyelvkészítés alapjait.

A BF működési elve egy, a Turing-géphez hasonló, cellákra felosztott szalag, mely egyszerre a program által felhasznált adatokat tartalmazza és egy külön lista, mely a program által végrehajtott utasításokból áll. Ezen kívül rendelkezik egy képzeletbeli „fejjel,” mely mindig az adatokat tároló szalag egyik cellájára mutat. Ez a fej felelős az adatok írásáért és olvasásáért, programozási analógiával élve a szalag egy hosszú, számokat tartalmazó tömb, a fej pedig a tömb egyik elemére mutató pointer.

A nyelv a következő utasításokat érti:

\begin{itemize}
    \item \textbf{<} - Egyel balra lépteti a fejet a szalagon.
    \item \textbf{>} - Egyel jobbra lépteti a fejet a szalagon.
    \item \textbf{+} - Egyel megnöveli a jelenlegi cella értékét.
    \item \textbf{-} - Egyel csökkenti a jelenlegi cella értékét.
    \item \textbf{[} - Ha a fej alatti érték nulla előreugrik a következő ] karakterhez az utasításokban. Ellenkező esetben szimplán tovább lép a következő utasításra. 
    \item \textbf{]} - Ha a fej alatti érték nem nulla, visszaugrik az előző [ karakterhez az utasításokban, különben tovább lép a következőre.
    \item \textbf{,} - Beolvas egy bájtot a bemenetről és eltárolja a szalag jelenlegi cellájában.
    \item \textbf{.} - Kiírja a fej alatti cella értékét a standard kimenetre karakterként értelmezve.
\end{itemize}

Egy egyszerű BF program, mely összead két számot: \verb|[->+<]|

Tehát addig hajtjuk végre ezeket az utasításokat, míg az első cellában lévő szám nullává nem válik úgy, hogy egyel csökkentjük az első cella értékét, majd jobbra lépünk, a jobboldalit egyel növeljük, majd visszalépünk az eredetire.

\subsubsection{Projekt bemutatása}

A nyelvre írt interpreter bemutatásához egy C nyelvben írt egyszerű interpretert[9] szeretnék felhasználni, mely emellett egy API-t is elérhetővé tesz a felhasználó számára, mellyel más projektekbe is beépíthetővé válik az értelmező.

A program belépés után ellenőrzi milyen közegben is lett meghívva és ennek alapján tölti be a kódot vagy egy fájlból vagy a standard bemenetről (vagy dob hibát hibás meghívás esetén.) Az egyszerűség kedvéért én a standard bemenetről történő beolvasást fogom végigvezetni, hisz a program többi funkcionalitása (fájlkezelés, szövegbeolvasás közben történő hibakezelés) számunkra nem releváns. A bemutatott kód egésze a 4.1-es forráskódban látható.

Az első sor inicializál egy üres láncolt listát, mely az utasításokat fogja tartalmazni. A második pedig egy üres 30.000 hosszúságú tömböt inicializál, ez lesz az szalagunk. 

Ezután belépünk a brainfuck\_parse\_string függvénybe, mely egyszerre végzi a Lexer és a Parser feladatát. A függvény átlépked az utasításokon és úgy csoportosítja a jobbra -balra lépéseket, a beolvasást és kiírást és a növelést-csökkenést, hogy az ugyanolyan fajtájúak egy-egy instrukcióvá változnak. Például, ha egymás után három pluszjelet lát a program, akkor ahelyett, hogy a szalag értékét háromszor egyel növelné, helyette egy olyan instrukciót helyez el, mely hárommal növeli az értéket egyszer. Ezen kívül, ami érdekes még, az az, hogy a szögletes zárójelek által definiált ciklusok külön beágyazott fákként jelennek meg a végső szintaxisfában.

Ezután az így kapott fát behelyezzük az utasításokat tartalmazó listába és lefuttatjuk. Mivel a BF utasításai remekül illeszkednek az egyszerű tömb/pointer műveletekre így a legtöbb elem szimplán egy sorrá átalakítható.

Végül pedig felszabadítjuk mind az utasítás, mind az adatok listájának memóriáját és sikerrel visszatérünk.

\subsection{Rockstar}

\subsubsection{A nyelvről röviden}

A Rockstar\cite{rockstar} egy paródia programozási nyelv, mely különlegessége, hogy a nyelven írt kód az elmúlt évtizedek Rock és Metál szövegeire emlékeztet. A nyelv a BF-hez hasonlóan szintén Turing-teljes, bár annak ellenére, hogy nem általános programozásra alkották sokkal inkább használható. A nyelvet Dylan Beattie alkotta, akit egy Twitteren olvasott üzenet ihletett meg.

A nyelv specifikációja megtalálható a Github tárolóján1, így én nem szeretnék erre külön kitérni, kivéve, ahol ez szükséges. Én a nyelv referencia implementációjára szeretnék most fókuszálni, mely egy JavaScriptben írt értelmező, mely a böngészőn belül és azon kívül is működik.

\subsubsection{Projekt bemutatása}

A referencia implementáció, mely a Satriani névre hallgat egy PEG.js nevű könyvtárat használ a Parser legenerálására.

„A PEG.js egy egyszerű JavaScript parser generátor, mely gyors parsereket generál remek hibaüzenetekkel. Bonyolult adatok feldolgozására, programozási nyelvekhez, fordítói rendszerekhez, értelmezőkhöz, fordítókhoz és egyéb eszközökhöz is egyszerűen használható.”[8]

Az ebből kapott szintaxisfát egy aránylag szimpla virtuális gép futtatja, mely két fő részből áll: Egy Environment-nek nevezett adatstruktúrából, mely a változókat és az egyéb, futtatáshoz szükséges, információkat tartalmazza és egy hosszú switch-elágazásból, mely végrehajtja a szintaxisfa utasításait.