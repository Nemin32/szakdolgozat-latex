A dolgozatom keretei között először is bevezettem a Pszeudokódot, mint programozási nyelvet, megadvan annak tulajdonságait, szintaktikáját és szemantikáját, kitérve külön az olyan elemekre, melyek a nyelvet definiáló jegyzettől eltérnek.

Ezután bemutatásra került annak folyamata, hogy a felhasználó által szolgáltatott nyers szövegen milyen átalakításokat is végzünk ahhoz, hogy az futtatható állapotba kerüljön. Bemutattam az egyes lépések módszereit és informális működésüket.

A felhasznált technikák felsorolása után után két nyelven keresztül mutattam be a programozási nyelvek fejlesztésének folyamatát. Az első példában egy olyan értelmezőt mutattam be, mely az egyik legegyszerűbb nyelvet képes lefuttatni, mely ennek ellenére Turing-teljes. A másodikban pedig egy olyan nyelvet, mely alkotásának folyamata nagyban informálta a saját programom.

Az ezt követő fejezetben mérlegeltem azokat a nyelveket, melyekre a projektem alapozhattam volna. Sorra vettem, hogy az egyes nyelveket, erősségeik és gyengeségeiket figyelembe véve miért nem választottam. Végül pedig hasonló módon bemutatásra került az értelmező alapjául szolgáló TypeScript nyelv.

A nyelv kiválasztását a rendszerterv követte, mely során felvázoltam a programom különálló részeit, ezek milyen kapcsolatban állnak egymással és, hogy a felhasználó bemenete és az ezekből generált kód milyen utat jár be sikeres és sikertelen fordítás esetén. Ezen kívül bemutatásra kerül még, hogy a felhasználói felület hogyan néz ki és miket kell, hogy tudjon.

Ezután a rendszertervben említett részek részletes bemutatására került sor. Definiáltam a \ref{sec:similar} fejezetben felsorolt lépések a dolgozatban elkészített projektre specifikus algoritmusait és technikáit, beleértve a tokenizálás folyamatát, a tokenekből szintaxisfát alkotó parsert, az ezt lineáris kóddá bontó fordítót, a program által használt virtuális gépet, és ennek utasításait táblázatos formában. Végül a felhasználói felület végleges dizájnja és funkcionalitása is bemutatásra került.

Az elkészült program ezután tesztelésre került kétféle módon. Az első egy kevésbé specifikus, de az összes jegyzetben található algoritmust ellenőrző módszer, mely a fordítás első négy lépését hajta végre, ellenőrizve, hogy a program konzisztens típusosság szempontjából, mely tulajdonság garantálja, hogy szintaktikailag és szemantikailag is helyes. A második módszer a fordítást elejétől a végéig ellenőrzi azzal, hogy néhány adott algoritmust konkrét értékekre futtatunk le, majd az algoritmus végrehajtása után ellenőrizzük, hogy az így kapott érték megfelel-e annak, amire számítottunk.

Az így kapott program képes az Óbudai Egyetemen tanított Pszeudokódon írt kód értelmezésére, a kód típusosságának és szintaktikai helyességének ellenőrzésére, és a lefordított állomány futtatására oly módon, mely felfedi a felhasználó számára a futtatási környezet belső működését. Ezáltal a dolgozat főbb célkitűzései megvalósításra kerültek.
