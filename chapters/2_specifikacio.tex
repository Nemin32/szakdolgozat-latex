Közel minden magyar, programozást tanító egyetemen, így az Óbudai Egyetemen is megtalálható egy Pszeudokódnak nevezett[17], legtöbbször C-re hasonlító képzeletbeli programozási nyelv. Ezek célja az algoritmusok egységesített alakba való átültetése és ezáltal a megtanulásuk megkönnyebbítése. Emellett pedig az olyan alapvető fogalmak megtanítása is, mint hogy mit jelent és miből áll a szintaxis, mik a típusok, mik a programok alkotóelemei és egyéb hasonlóképp kihagyhatatlan részei a programozó eszköztárának. Ebben a fejezetben specifikálni szeretném pontosan milyen elemekből is áll a Pszeudokód, milyen szintaxissal rendelkezik és milyen egyéb igényeket támaszt, melyet a dolgozatomnak el kell tudnia látni.

\subsection{Turing teljesség}
„A Turing-teljesség egy programozási- vagy lekérési nyelv vagy bármely más számítási modell erejéről szóló állítás, mely kijelenti, hogy bármit, amit egy Turing-géppel kiszámoltathatunk ezzel a nyelvvel vagy számítási modellel is ki lehet. Mivel a Turing-gépek a legerősebb eddig ismert számítási modell, így a Turing teljesség azt jelenti, hogy bármi ami egyáltalán kiszámítható, kiszámítható a nyelv használatával is. Ez a programozási nyelvek számára igencsak kívánatos tulajdonság, mivel ez azt jelenti, hogy a nyelv nem korlátozza a felhasználót abban, milyen problémát is oldhat meg vagy számítást fejezhet ki benne.”[6]

Ez a tulajdonság még azért is lényeges, mivel az ilyen nyelvekben leírt kód bizonyítottan bármely más Turing-teljes nyelvre átírható. Ahhoz, hogy a teljességet bizonyítsuk, meg kell mutatnunk, hogy a nyelv képes egy univerzális Turing-gépet szimulálni. Ez a képzeletbeli gép egy végtelen hosszú tekerccsel rendelkező olvasó/író fej, mely egy véges hosszúságú utasítási lista alapján képes a tekercsről adatokat beolvasni, majd ezeket értelmezve új adatokat írni arra (vagy a meglévőket felülírni.)

Mivel a Pszeudokód képes (elvben) tetszőleges hosszúságú tömbök létrehozására, ezekben a tömbökben véletlenszerű adatelérésre, módosításra és rendelkezik folyamatvezérlő szerkezetekkel, így kimondhatjuk, hogy a nyelv Turing-teljes és ezáltal bármely más tetszőleges ennek megfelelő nyelv kódját átfordíthatjuk rá vagy bármely Pszeudokódban írt program átfordítható egy másik ilyen nyelvre.

\subsection{Szintaktikai elemek és szemantikájuk}

A Pszeudokód sokat merít a C-szerű nyelvekből, így sok hasonlóság is felfedezhető benne. A következőkben a Pszeudokód szemantikai elemei és értelmezésük kerül bemutatásra.

\begin{code}{A Pszeudokód szintaktikai specifikációja}{code:syntax}
expression ::= arrayIndex | atom | binaryOperation | functionCall | not | reference | variable

statement ::= arrayComprehension | assignment | block | debug | for | functionCall | functionDeclaration | if | newArray | print | return | swap | while

block ::= statement+

newArray ::= variable " <- " (oneDimensionalArray | multiDimensionalArray)
multiDimensionalArray ::= "TáblaLétrehoz(" atomType ")[" expression ("," expression)+ "]"
oneDimensionalArray ::= "Létrehoz(" atomType ")[" expression "]"

swap ::= address " <- " address

if ::= ifHead (ifElseIf* ifElse)? " elágazás vége"

ifElseIf ::= ( " különben ha " expression " akkor " block)
ifElse ::= " különben " block
ifHead ::= "ha " expression " akkor " block 

for ::= "ciklus" variable "<-" expression ("-tól" | "-től") expression "-ig" block " ciklus vége"

while ::= normalWhile | doWhile
doWhile ::= "ciklus " block " amíg " expression
normalWhile ::= "ciklus amíg " expression block " ciklus vége"

return ::= "vissza " (expression | arrayComprehension) 

debug ::= "debug"

print ::= "kiír " expression

binaryOperation ::= logicOp

mulOp ::= primary | primary ("*" | "/" | "mod") primary
addOp ::= mulOp | mulOp ("+" | "-") mulOp
compOp ::= addOp | addOp ("<" | "<=" | ">" | ">=" | "=" | "=/=") addOp
logicOp ::= compOp | compOp ("és" | "vagy") compOp

primary ::= "(" expression ")" | not | reference | functionCall | arrayIndex | variable | atom

not ::= "~" expression
reference ::= "&" address

functionCall ::= functionName ("()" | "(" expression ("," expression)* ")")

functionDeclaration ::= "függvény " functionName ("()" | "(" parameter ("," parameter)* ")") statement+ " függvény vége"

functionName ::= [A-Z] ([a-z] | [A-Z])*

parameter ::= "címszerint "? (functionName | variable) " : " type

baseType ::= atomType | arrayType
arrayType ::= "rendezett "? atomType " tömb"
atomType ::= "egész" | "szöveg" | "logikai"


address ::= arrayIndex | variable

arrayIndex ::= variable "[" expression ("," expression)* "]"

variable ::= [a-z] ([a-z] | [A-Z])+

atom ::= string | number | boolean

string ::= '"' [a-z]* '"'
number ::= "0" | ( "-"? [1-9] [0-9]* )
boolean ::= "Igaz" | "igaz" | "Hamis" | "hamis"

nl ::= " "* "\n"+ " "*

\end{code}

\subsection{Függvények és eljárások}

A Pszeudokód különbséget tesz a függvények és eljárások között. Az előbbi visszatér értékkel, az utóbbi nem. Mindkettő a 1. kódnál szereplő szignatúrát követi:

Az Óbudai Egyetemen használt Pszeudokódok része, hogy a sorok elején az olvasás és idézés egyszerűbbé tételének kedvéért számokat helyezett el az író.

Ezt pusztán az olvasó számára szolgál valós információval, a nyelv önmaga nem rendelkezik se GOTO utasítással, se más egyéb szerkezettel mellyel sorszámalapú ugrást tudnánk végrehajtani, így az én megvalósításomban is ezek figyelmen kívül lesznek hagyva.

\subsection{Kimenet/Bemenet sorok}

Minden Pszeudokód algoritmus első két sora arra való, hogy definiálja a be- és kimeneti változókat, mellyel az algoritmus dolgozik. Ez a 2.2-es forráskódban látható.

Mivel mind a bemeneti, mind a kimeneti változók kiderülnek a függvény szignatúrájából és visszatérési értékéből, így az én megvalósításomban ezek figyelmen kívül lesznek hagyva.

\subsection{Visszatérési értékek}

A Pszeudokód a C-szerű nyelvekkel ellentétben támogatja a több értékkel való visszatérést. Tehát egy érték helyett, akár kettővel, hárommal vagy elméletileg többel is visszatérhetünk a függvényekből.

A helyzetet még az komplikálja, hogy bizonyos esetekben más számú visszatérési értékekkel kell számolnunk. Ez legtöbbször például a kereső függvényeknél van használva, ahol az első érték azt jelzi, hogy egyáltalán megtaláltuk-e a keresett értéket, míg a második ennek indexével tér vissza, ha igen. Ilyen esetekben, ha nem találtunk megfelelő értéket egy darab visszatérési értékkel rendelkezünk, ha igen, akkor pedig kettővel.

Mivel a Pszeudokód támogatja a tömböket, így több értékkel való visszatérés esetén egyszerűen egy tömböt adunk vissza.