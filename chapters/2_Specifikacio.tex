Közel minden magyar, programozást tanító egyetemen, így az Óbudai Egyetemen is megtalálható egy Pszeudokódnak nevezett, legtöbbször C-re hasonlító képzeletbeli programozási nyelv. 

Lényegük, hogy az informálisan definiált algoritmusokat, mint például az "adjuk össze az első tíz páros számot" a \ref{code:firstten} forráskódhoz hasonló egységes, a számítógép által is értelmezhető utasításokra fordítható alakra hozza:

\begin{code}{Az első tíz páros szám összege}{code:firstten}
i <- 1
számláló <- 0
összeg <- 0

ciklus amíg számláló < 10
    ha i mod 2 = 0 akkor
        számláló <- számláló + 1
        összeg <- i
    elágazás vége

    i <- i + 1
ciklus vége

kiír összeg
\end{code}

Ennek célja az algoritmikus gondolkodás elsajátításának megkönnyebbítése, ezen kívül az olyan fogalmak bemutatása és megtanítása is, mint hogy mit jelent és miből áll a szintaxis, mik a típusok, az alapvető alkotóelemek, melyek minden programban megtalálhatóak, és egyéb, hasonlóképp kihagyhatatlan részei a programozó eszköztárának.

Ebben a fejezetben specifikálni szeretném pontosan milyen elemekből is áll a Pszeudokód, milyen szintaxissal rendelkezik és milyen egyéb igényeket támaszt, melyet a dolgozatomnak el kell tudnia látni.

\subsection{Bevezetés}

A Pszuedokód egy imperatív, Turing-teljes programozási nyelv, mely a C-szerű nyelvek családjába tartozik.

Az imperatív nyelvek esetében a programozó egy bizonyos szintű abszrtraktció mellett a "Hogyan?" kérdésre válaszolva alkotja meg a programját. Ez alatt azt értjük, hogy a gép számára egzakt utasításokat adunk, mint például "töltsd be a \texttt{3} értéket az \texttt{i} változóba" vagy "ha \texttt{x} érték igaz, ugorj az \texttt{igaz\_ág} elágazásba". A legtöbb C-szerű nyelv, beleértve a Pszeudokódot is, imperatív.

Velük ellentétben állnak az úgynevezett deklaratív nyelvek, melyeknél a programozó a "Mit?" kérdésre ad választ. Ilyen például az SQL nyelv, ahol nem feladtunk meghatázorzni pontosan hogyan is érjük el az egyes elemeket egy adatbázisból, csupán annak deklarálása hogy melyik elemeket szeretnénk lekérni.

A Turing-teljesség definícióját Stephan Kepser cikkéből kölcsönöztem, melyben XSLT nyelvre bizonyítja a tulajdonság fennállását:

\say{\textit{A Turing-teljesség egy programozási- vagy lekérési nyelv vagy bármely más számítási modell erejéről szóló állítás, mely kijelenti, hogy bármit, amit egy Turing-géppel kiszámoltathatunk ezzel a nyelvvel vagy számítási modellel is ki lehet. Mivel a Turing-gépek a legerősebb eddig ismert számítási modell, így a Turing teljesség azt jelenti, hogy bármi ami egyáltalán kiszámítható, kiszámítható a nyelv használatával is. Ez a programozási nyelvek számára igencsak kívánatos tulajdonság, mivel ez azt jelenti, hogy a nyelv nem korlátozza a felhasználót abban, milyen problémát is oldhat meg vagy számítást fejezhet ki benne.}}\cite{turing}

Ahhoz, hogy a teljességet bizonyítsuk, meg kell mutatnunk, hogy a nyelv képes egy univerzális Turing-gépet szimulálni. Ez a képzeletbeli gép egy végtelen hosszú tekerccsel rendelkező olvasó/író fej, mely egy véges hosszúságú utasítási lista alapján képes a tekercsről adatokat beolvasni, majd ezeket értelmezve új adatokat írni arra (vagy a meglévőket felülírni.)

Mivel a Pszeudokód képes (elvben) tetszőleges hosszúságú tömbök létrehozására, ezekben a tömbökben véletlenszerű adatelérésre és módosításra, és rendelkezik folyamatvezérlő szerkezetekkel, így kimondhatjuk, hogy a nyelv Turing-teljes és ezáltal bármely más tetszőleges ennek megfelelő nyelv kódját átfordíthatjuk rá vagy bármely Pszeudokódban írt program átfordítható egy másik ilyen nyelvre.

\subsection{Szintaktikai elemek és szemantikájuk}

A Pszeudokód sokat merít a C-szerű nyelvekből, így sok hasonlóság is felfedezhető benne, ám bizonyos részeiben eltér. A következőkben a Pszeudokód szemantikai elemei és értelmezésük kerül bemutatásra.

Az egyes programozási nyelvek "nyelvtana" (angol szakszóval a nyelv \textit{"grammar"}-ja), alatt azokat a szabályokat értjük, melyek az összes a nyelv által értelmezett, szintaktikailag érvényes programot jellemzik. Ide értjük az összes kulcsszó és operátorok felsorolását, a nyelv által használt írásjeleket, és azt is, hogy ezeket milyen alakban vagyunk képesek összeépíteni.

John W. Backus 1959-ben az ALGOL 60 nyelv nyelvi elemeinek formális definiálására alkotott meg egy leíró nyelvet\cite{bnf}, melyet ma (Peter Naur hozzájárulásainak köszönhetően, aki Backus eredeti nyelvét egyszerűbb alakba hozta) Backus-Naur-formaként (BNF) ismerünk. A BNF segítségével konzisztens és formális alakban tudjuk rögzíteni a nyelveket definiáló, fentebb említett elemeket.

\begin{code}{Példa a Backus-Naur-formára}{code:bnfexample}
<non-zero> ::= '1' | '2' | '3' 
            | '4' | '5' | '6' 
            | '7' | '8' | '9'

<zero> ::= '0'

<digit> ::= <zero> | <non-zero>

<number-tail> ::= <digit> | <digit> <number-tail>

<number> ::= <digit> | <nonZero> <number-tail> 
\end{code}

A \ref{code:bnfexample} forráskód bemutatja, hogyan is lehet BNF alakba hozni a pozitív, egész számokat, a nullát belevéve: 

\begin{itemize}
  \item \texttt{<non-zero>} - először definiáljuk a nem-nulla számjegyeket,
  \item \texttt{<zero>} - majd a nullát, 
  \item \texttt{<digit>} - eztán a számjegy fogalmát, mely lehet nem-nulla számjegy vagy nulla,
  \item \texttt{<number-tail>} - rekurzívan adjuk meg, hogy egy szám a második számjegytől kezdve lehet vagy egy számjegy önmagában, vagy egy számjegy melyet a szám további számjegyei követnek. Erre a szabályra azért van külön szükség, hogy elkerüljük a "0123" alakban megadott számokat, melyeket nem értelmezünk.
  \item \texttt{<number>} - végül pedig kijelentjük, hogy egy szám az vagy egy számjegy, vagy pedig egy nem-nulla számjegy melyet tetszőleges számú (de legalább egy) más számjegy követ.
\end{itemize}

Ahogy az ebből az egyszerű példából is látható, önmagában a BNF alak kissé esetlen. Olyan intuitív fogalmak, mint az opcionális elemek (tehát olyan elemek, melyek hiánya esetén is érvényes a szabály) és az egymást követő karakterek rövidített leírása (tehát, ha azt szeretnénk, hogy egy szabály például az angol ABC összes betűjére vonatkozzon, ahelyett, hogy azt mondhatnánk, hogy "a-z", mind a huszonhat betűt ki kéne írnunk) hiányoznak belőle, így több programozói és szabványkészítő csoport amellett döntött, hogy kiterjeszti a formátumot.

Ezek közül én a World Wide Web Consortium által, eredetileg az XML nyelv leírásához kifejlesztett\cite{w3c}, \textit{Extended Brackus-Naur Form} (EBNF)\footnote{Valójában mind a BNF, mind az EBNF nem önálló nyelvek, hanem hasonló vonásokkal rendelkező nyelvcsaládok, ezáltal ugyanazt a nevet több szabvány is viseli. EBNF-nek nevezzük többek között az ISO/IEC 14977\cite{isoebnf} szabványt is, mely szintén a BNF kiterjesztése, ám a W3C által specifikált nyelvhez képest eltérő szintaxissal rendelkezik. Az egyszerűség kedvéért dolgozatomban úgy hivatkozom az utóbbi csoport által fejlesztett változatra, mintha önálló nyelv lenne.}, vagyis Kiterjesztett Brackus-Naur Formát kívánom használni a Pszeudokód leírására. Ez lehetővé teszi olyan szabályok egyszerű leírását, melyek a \ref{sec:parser} alfejezetben használt értelmező több művöletet egy-az-egyben mintázza.

\begin{code}{A Pszeudokód szintaktikai specifikációja}{code:syntax}
expression ::= arrayIndex | atom | binaryOperation | functionCall | not | reference | variable

statement ::= arrayComprehension | assignment | block | debug | for | functionCall | functionDeclaration | if | newArray | print | return | swap | while

block ::= statement+

newArray ::= variable " <- " (oneDimensionalArray | multiDimensionalArray)
multiDimensionalArray ::= "TáblaLétrehoz(" atomType ")[" expression ("," expression)+ "]"
oneDimensionalArray ::= "Létrehoz(" atomType ")[" expression "]"

swap ::= address " <- " address

if ::= ifHead (ifElseIf* ifElse)? " elágazás vége"

ifElseIf ::= ( " különben ha " expression " akkor " block)
ifElse ::= " különben " block
ifHead ::= "ha " expression " akkor " block 

for ::= "ciklus" variable "<-" expression ("-tól" | "-től") expression "-ig" block " ciklus vége"

while ::= normalWhile | doWhile
doWhile ::= "ciklus " block " amíg " expression
normalWhile ::= "ciklus amíg " expression block " ciklus vége"

return ::= "vissza " (expression | arrayComprehension) 

debug ::= "debug"

print ::= "kiír " expression

binaryOperation ::= logicOp

mulOp ::= primary | primary ("*" | "/" | "mod") primary
addOp ::= mulOp | mulOp ("+" | "-") mulOp
compOp ::= addOp | addOp ("<" | "<=" | ">" | ">=" | "=" | "=/=") addOp
logicOp ::= compOp | compOp ("és" | "vagy") compOp

primary ::= "(" expression ")" | not | reference | functionCall | arrayIndex | variable | atom

not ::= "~" expression
reference ::= "&" address

functionCall ::= functionName ("()" | "(" expression ("," expression)* ")")

functionDeclaration ::= "függvény " functionName ("()" | "(" parameter ("," parameter)* ")") statement+ " függvény vége"

functionName ::= [A-Z] ([a-z] | [A-Z])*

parameter ::= "címszerint "? (functionName | variable) " : " type

baseType ::= atomType | arrayType
arrayType ::= "rendezett "? atomType " tömb"
atomType ::= "egész" | "szöveg" | "logikai"


address ::= arrayIndex | variable

arrayIndex ::= variable "[" expression ("," expression)* "]"

variable ::= [a-z] ([a-z] | [A-Z])+

atom ::= string | number | boolean

string ::= '"' [a-z]* '"'
number ::= "0" | ( "-"? [1-9] [0-9]* )
boolean ::= "Igaz" | "igaz" | "Hamis" | "hamis"

nl ::= " "* "\n"+ " "*

\end{code}

\subsection{Eltérések a jegyzettől}

Mivel a szakdolgozat célja a Szoftvertervezés és Fejlesztés I. hallgatóinak megsegítése, így igyekeztem amennyire csak lehet hűen adaptálni az órán használt jegyzetet\cite{jegyzet}.

Ennek ellenére van egy-két részlet, amiben célszerűnek láttam eltérni a jegyzetben látottaktól. A következő fejezetekben ezeket veszem sorra.

\subsubsection{Sorszámozás}

Az jegyzetben használt Pszeudokód algoritmusainak része, hogy a sorok elején az olvasás és idézés egyszerűbbé tételének kedvéért sorszámokat helyezett el az író. Ez a \ref{fig:line_number} képen látható.

\addimage{sorszam.png}{Példa az algoritmusok sorszámozására}{line_number}

Ezt pusztán az olvasó számára szolgál valós információval, a nyelv önmaga nem rendelkezik se GOTO utasítással, se más egyéb szerkezettel mellyel sorszám alapú ugrást tudnánk végrehajtani, így az én megvalósításomban ezek nem kerültek felhasználásra.

\subsubsection{Függvények szignatúrája}

Az egyes algoritmusok a lentebb tárgyalt Kimenet sor miatt nem nyilatkoznak arról milyen típusú értékkel is térnek vissza. Ez az esetek nagy többségében nem jelent problémát, hisz az értelmező a függvénytörzsben felhasznált értékek típusait követve ki tudja következtetni a visszatérési érték típusát is.

Azonban olyan esetben amikor a függvény rekurzívan meghívja önmagát, erre nincs lehetőségünk, így ilyenkor a függvény deklaráció sorában azt is expliciten meg kell mondanunk, a függvény milyen értékkel is tér majd vissza:

\begin{code}{Példa a rekurzív függvények szignatúrájára}{code:signature}
függvény FibonacciRekurzív(n : egész): egész
  ha n <= 1 akkor
    vissza 1
  különben
    vissza FibonacciRekurzív(n - 2) + FibonacciRekurzív(n - 1)
  elágazás vége
függvény vége

kiír FibonacciRekurzív(5)
\end{code}

Ahogy az a \ref{code:signature} forráskódban látható, a függvény paramétereinek listája után a kettőspont, majd "\texttt{egész}" típus szignatúrával jelöljük, hogy a függvény végeredménye egy szám érték lesz.

\subsubsection{Kimenet/Bemenet sorok}

Minden Pszeudokód algoritmus első két sora arra való, hogy definiálja a be- és kimeneti változókat, mellyel az algoritmus dolgozik. Ez a \ref{fig:inputoutput} képen látható.

\addimage{bemkim.png}{Példa a Bemenet és Kimenet sorokra}{inputoutput}

Mivel mind a bemeneti változók, mind a kimeneti érték vagy értékek kiderülnek a függvény szignatúrájából és visszatérési értékéből, így az én megvalósításomban ezek nem kerültek felhasználásra.

\subsubsection{Változó számú visszatérési érték}

A Pszeudokód a C-szerű nyelvekkel ellentétben támogatja a több értékkel való visszatérést. Tehát egy érték helyett, akár kettővel, hárommal vagy elméletileg akár többel is visszatérhetünk a függvényekből.

A helyzetet még az komplikálja, hogy bizonyos esetekben más számú visszatérési értékekkel kell számolnunk. Ez legtöbbször például a kereső függvényeknél van használva, ahol az első érték azt jelzi, hogy egyáltalán megtaláltuk-e a keresett értéket, míg a második ennek indexével tér vissza, ha igen. Ilyen esetekben, ha nem találtunk megfelelő értéket egy darab visszatérési értékkel rendelkezünk, ha igen, akkor pedig kettővel.

Ennek támogatása erősen bonyolítana a típusellenörző kódján, így helyette kötelezem a függvényeket, hogy minden águkban ugyanolyan típusú és mennyiségű értékkel térjenek vissza. Például a fentebb említett függvényeknél, ha nem találjuk a keresett értéket \texttt{(hamis, 0)} elempárral térünk vissza.

\subsubsection{Különleges karakterek}

A jegyzetben található Pszeudokód teljes mértékben igénybe veszi az UTF-8 által nyújtott matematikai szimbólumokat a változók értékének hozzárendeléséhez használt nyíltól kezdve a halmazműveletekig.

Ezek begépelése nem egyszerű az átlag felhasználó számára, így úgy döntöttem az egyszerűbb jeleket ASCII megfelelőkkel (például a nyíl helyett "\texttt{<-}", az áthúzott egyenlőség jel helyett "\texttt{=/=}"), az ennél különlegesebb szimbólumokat pedig azok kiírt neveivel helyettesítem (például "\texttt{végtelen}" vagy "\texttt{Halmaz}").